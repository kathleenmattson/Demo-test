\documentclass[]{scrartcl}
\usepackage[left=25mm,right=25mm,top=30mm,bottom=30mm,head=24pt]{geometry}
%\usepackage{environ} % for conditional text
%\usepackage{etoolbox} % for conditional text
%\usepackage{xcolor}
%\usepackage{grffile} % image support (like graphicx)
\usepackage{float} % for positioning figures
%\usepackage{longtable}
\usepackage{supertabular}
\usepackage{booktabs} % booktabs table sytles
\usepackage{makecell} % additional table features + \Xhline
\usepackage{ragged2e}
%\usepackage{etoolbox}  % supports conditional tests

\KOMAoption{parskip}{full} % change default spacing between paragraphs
\hyphenpenalty=500
\tolerance=2000
\emergencystretch=10pt

\KOMAoption{fontsize}{10pt}
\usepackage{fontspec}
\usepackage{lmodern}            % Latin Modern Font with lots of glyphs
%\setmainfont{}   % any serif font on the system (times roman)
%\setsansfont{}   % any sans-serif font on the system ()
%\setmonofont{}   % any monospace font on the system ()
\usepackage{microtype}
\raggedright

\usepackage{amsmath}
\usepackage{amssymb}
\usepackage{amsthm}

% Moved
\usepackage{hyperref} % Clickable links with \ref{} and \href{}
\hypersetup{
	colorlinks=true,
	linkcolor=blue,
	filecolor=magenta,      
	urlcolor=cyan,
	pdfpagemode=FullScreen,
}

\usepackage{grffile} % image support (like graphicx). Usage: \includegraphics{path/to/file}

% Define \thickhline, a variant of \hline
\newcommand{\thickhline}{\Xhline{0.3ex}} % requires package makecell

\begin{document}
	
\section{Tables}

Online \LaTeX table generator: \href{https://www.tablesgenerator.com/}{https://www.tablesgenerator.com/}.

\subsection{Simple Table, Centered Table, Left-Justified Data}

\begin{table}[h!]
	\centering
	\begin{tabular}{ll}
		column 1 & column 2 \\
		\hline
		celery & apples \\
		milk & peanut butter 
	\end{tabular}
	\caption{Table centered, data left}\label{tab:center-left}
\end{table}

\subsection{Simple Table, Booktabs Style}

%\usepackage{booktabs}  % goes in the main .tex file preamble, not here

\begin{table}[h!]
	\centering
	\begin{tabular}{@{}ll@{}}
		\toprule
		column 1 & column 2 \\
		\hline
		celery & apples \\
		milk & peanut butter 
	\end{tabular}
	\caption{Table centered, data left}\label{tab:booktabs}
\end{table}

\subsection{Simple Table, Centered Table, Centered Data}

\begin{table}[h!]
	\centering
	\begin{tabular}{cc}
		column 1 & column 2 \\
		\hline
		celery & apples \\
		milk & peanut butter 
	\end{tabular}
	\caption{Table centered, data centered}\label{tab:center-center}
\end{table}

\subsection{Simple Table, Float Left, Left Data}

\begin{table}[h!]
	\begin{tabular}{ll}
		column 1 & column 2 \\
		\hline
		celery & apples \\
		milk & peanut butter 
	\end{tabular}
	\caption{Table left, data left}\label{tab:left-left}
\end{table}

\subsection{Wrap Text in Cells}

\begin{table}[h!]
	\centering
	\begin{tabular}{lp{.8\textwidth}}
		column 1 & column 2 \\
		\hline
		celery & Lorem ipsum dolor sit amet, consectetur adipiscing elit, sed do eiusmod tempor incididunt ut labore et dolore magna aliqua. Ut enim ad minim veniam, quis nostrud exercitation ullamco laboris nisi ut aliquip ex ea commodo consequat. \\
		milk & peanut butter 
	\end{tabular}
	\caption{Wrapped text}\label{tab:wrapped-text}
\end{table}

\subsection{Unnumbered Table Without Caption}

\begin{center}
\begin{tabular}{ll}
	column 1 & column 2 \\
	\hline
	celery & apples \\
	milk & peanut butter 
\end{tabular}
\end{center}

\subsection{Extra Horizontal Padding}

\begin{table}[h!]
	\centering
	\setlength{\tabcolsep}{3.5em}
	\begin{tabular}{ll}
		column 1 & column 2 \\
		\hline
		celery & apples \\
		milk & peanut butter 
	\end{tabular}
	\caption{Increase horizontal cell padding}\label{tab:horiz-padding}
\end{table}

\subsection{Extra Vertical Spacing}

\begin{table}[h!]
	\renewcommand{\arraystretch}{1.5}
	\centering
	\begin{tabular}{ll}
		column 1 & column 2 \\
		\hline
		celery & apples \\
		milk & peanut butter 
	\end{tabular}
	\caption{Extra line spacing}\label{tab:line-spacing}
\end{table}

\subsection{Table with Cell Borders}

\begin{table}[h!]
	\renewcommand{\arraystretch}{1.2}
	\centering
	\begin{tabular}{|l|l|}
		\hline 
		Column 1 & Column 2 \\
		\hline
		celery & apples \\
		\hline
		milk & peanut butter \\
		\hline
	\end{tabular}
	\caption{Table with cell borders}\label{tab:all-borders}
\end{table}

\end{document}
